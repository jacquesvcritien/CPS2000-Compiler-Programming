			\documentclass{article}
			\usepackage[margin=1.25in]{geometry}
			\usepackage{graphicx}
			\usepackage{listings}
			\usepackage{titlepic}
			\usepackage{float}
			\usepackage{hyperref}
\usepackage{xcolor}
			\newcommand\tab[1][1cm]{\hspace*{#1}}
			\usepackage{multicol}
			\renewcommand{\labelenumii}{\theenumii}
			\newcommand{\quotes}[1]{``#1''}
			\renewcommand{\theenumii}{\theenumi.\arabic{enumii}.}
			
			\begin{document}
			\title{\includegraphics[scale = .6]{uom.png}
				\linebreak 
				\textbf{CPS2000 - Compiler Theory \& Practise}\linebreak\linebreak
				\textbf{Assignment Part 2}\linebreak\linebreak
				\large{B.Sc Computer Science}
				\date{}
				\author{Jacques Vella Critien - 97500L}}
				
				\begin{titlepage}
					\maketitle
					\thispagestyle{empty}
				\end{titlepage}
				
				\tableofcontents
				\newpage
				
				\section{Task1: Extending SmallLang}
				
				For the first task of this part of the assignment, we were required to extend SmallLang into SmallLangV2 by adding some other features. These features include adding support for the primitive type \quotes{char} and for arrays which hold a series of elements of the same type in contiguous memory. It was required to let array values uninitialised by default but an implementation for initialisation for values was also required. Moreover, formal parameters had to be changed in order to support both the \quotes{char} type and the arrays as types. In order to implement this, as can be seen below, EBNF rules had to be added and some were changed.
				
				
				
				
		
		\bibliographystyle{ieeetr}
		\nocite{*}
\bibliography{references2}
			
		
			
					
			\end{document}